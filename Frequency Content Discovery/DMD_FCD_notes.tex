\documentclass[11pt,reqno]{amsart}
\usepackage[margin=0.5in]{geometry}                % See geometry.pdf to learn the layout options. There are lots.
\geometry{letterpaper}                   % ... or a4paper or a5paper or ... 
%\geometry{landscape}                % Activate for for rotated page geometry
%\usepackage[parfill]{parskip}    % Activate to begin paragraphs with an empty line rather than an indent
\usepackage{graphicx}
\usepackage{amssymb}
\usepackage[all]{xy}
\usepackage{epstopdf}
\usepackage{color}
\DeclareGraphicsRule{.tif}{png}{.png}{`convert #1 `dirname #1`/`basename #1 .tif`.png}

%\graphicspath{{./Figs/}}
\graphicspath{{.}}

% these packages make it easy to include figures in the text. 
\usepackage{float}
\restylefloat{figure}
% newcommand

\begin{document}
\section{Varied Window Size}
\begin{itemize}
\item
Data with frequency content from $\omega = 2$ to $\omega = 100$ (in equal-frequency pairs to allow for DMD complex conjugate pair modes)
\item
Vary the size of the window being dragged across for DMD
\item
Result: All window sizes seem to pick up on all frequencies just fine (granted this data is 100\% sinusoidal with no noise)
\end{itemize}

\begin{figure}[htb!]
\includegraphics[width=\linewidth]{./Figs/Varied_Window/pairwise_wSizes.png}
\end{figure}
\begin{figure}[htb!]
\includegraphics[width=\linewidth]{./Figs/Varied_Window/pairwise_Fig1.png}
\end{figure}
\begin{figure}[htb!]
\includegraphics[width=\linewidth]{./Figs/Varied_Window/pairwise_Fig2.png}
\end{figure}
\begin{figure}[htb!]
\includegraphics[width=\linewidth]{./Figs/Varied_Window/pairwise_Fig3.png}
\end{figure}
\begin{figure}[htb!]
\includegraphics[width=\linewidth]{./Figs/Varied_Window/pairwise_Fig4.png}
\end{figure}
\begin{figure}[htb!]
\includegraphics[width=\linewidth]{./Figs/Varied_Window/pairwise_Fig5.png}
\end{figure}
\begin{figure}[htb!]
\includegraphics[width=\linewidth]{./Figs/Varied_Window/pairwise_Fig6.png}
\end{figure}

\newpage
\section{Degenerate Freqency Splitting}
\begin{itemize}
\item
Data consisting of 100 sinusoidal oscillators
\item
Amplitude + phase randomized
\item
50 different frequencies (2, 4, 6, ... 100) with 2 oscillators each
\item
Run MW-DMD on this data set untouched, and then gradually introduce splitting of the degenerate pairs until frequencies are all idstinct and evenly spaced (1, 2, 3, ... 100)
\item
Results: Rank-100 DMD on 100 evenly-spaced frequencies fails miserably. DMD relies on complex-conjugate pairs of modes to reconstruct an all-real input signal. With all frequencies evenly spaced DMD yields only real mode frequencies
\end{itemize}


\begin{figure}[htb!]
\includegraphics[width=\linewidth]{./Figs/Varied_Splitting/Fig1.png}
\end{figure}
\begin{figure}[htb!]
\includegraphics[width=\linewidth]{./Figs/Varied_Splitting/Fig2.png}
\end{figure}
\begin{figure}[htb!]
\includegraphics[width=\linewidth]{./Figs/Varied_Splitting/Fig3.png}
\end{figure}
\begin{figure}[htb!]
\includegraphics[width=\linewidth]{./Figs/Varied_Splitting/Fig4.png}
\end{figure}
\begin{figure}[htb!]
\includegraphics[width=\linewidth]{./Figs/Varied_Splitting/Fig5.png}
\end{figure}
\begin{figure}[htb!]
\includegraphics[width=\linewidth]{./Figs/Varied_Splitting/Fig6.png}
\end{figure}

\newpage
\section{Varied Rank}
\begin{itemize}
\item
Data set: 100 oscillators with pairwise frequencies of $\omega = 2, 4, 6,... 100$
\item
Rank varied from $r=100$ down to $r = 10$
\item
Complex $\omega$ spectra go from the true spectrum at $r=100$ (i.e. evenly spaced pure-imaginary frequencies) to being highly peaked about small, negative real frequencies
\item
Visualization of the DMD reconstructions at these ranks shows that low-frequency signals are lost first as rank drops
\item
In the low-rank limit, all imaginary $\omega$s are lost except one pair at $\omega \approx \pm 95i$

\end{itemize}
\begin{figure}[htb!]
\includegraphics[width=\linewidth]{./Figs/Varied_Rank/pairwise_Fig1.png}
\end{figure}
\begin{figure}[htb!]
\includegraphics[width=\linewidth]{./Figs/Varied_Rank/pairwise_Fig2.png}
\end{figure}
\begin{figure}[htb!]
\includegraphics[width=\linewidth]{./Figs/Varied_Rank/pairwise_Fig3.png}
\end{figure}
\begin{figure}[htb!]
\includegraphics[width=\linewidth]{./Figs/Varied_Rank/pairwise_Fig4.png}
\end{figure}
\begin{figure}[htb!]
\includegraphics[width=\linewidth]{./Figs/Varied_Rank/pairwise_Fig5.png}
\end{figure}
\begin{figure}[htb!]
\includegraphics[width=\linewidth]{./Figs/Varied_Rank/pairwise_Fig6.png}
\end{figure}
\begin{figure}[htb!]
\includegraphics[width=\linewidth]{./Figs/Varied_Rank/pairwise_Fig7.png}
\end{figure}
\begin{figure}[htb!]
\includegraphics[width=\linewidth]{./Figs/Varied_Rank/pairwise_Recons.png}
\end{figure}


\newpage
\section{Frequencies Sampled from Bimodal Distribution}
\begin{itemize}
\item
Each of the $N=100$ elements of the data set $X$ is a sum of sinusoids with frequencies from $\omega = 2$ to $\omega=100$ with amplitudes given by the function
\begin{equation}
\begin{split}
A(\omega) &\propto \sum_{i=1}^2\exp\left(-\frac{(\omega-\mu_i)^2}{\sigma_i^2}\right)
\end{split}
\end{equation}
This is a bimodal double-gaussian distribution with peaks at $\mu_1$ and $\mu_2$
\item
For these results, $\mu_1 = 24$, $\mu_2 = 61$, and $\sigma_1=\sigma_2=5$
\item
Results: At full-rank, the DMD frequency content is slightly peaked around the true frequency means. There is also a strong representation of modes at frequencies other than the true means. These erroneous modes appear exclusively at frequencies below $\mu_2$
\item
At rank 50 and below, frequency distribution stops being peaked about $\mu_1$ and $\mu_2$. By rank 25 high frequencies have been filtered out to the point that there aren't even any modes near $\mu_2$
\end{itemize}

\begin{figure}[htb!]
\includegraphics[width=\linewidth]{./Figs/Varied_Rank/bimodal_Fig1.png}
\end{figure}
\begin{figure}[htb!]
\includegraphics[width=\linewidth]{./Figs/Varied_Rank/bimodal_Fig2.png}
\end{figure}
\begin{figure}[htb!]
\includegraphics[width=\linewidth]{./Figs/Varied_Rank/bimodal_Fig3.png}
\end{figure}
\begin{figure}[htb!]
\includegraphics[width=\linewidth]{./Figs/Varied_Rank/bimodal_Fig4.png}
\end{figure}
\begin{figure}[htb!]
\includegraphics[width=\linewidth]{./Figs/Varied_Rank/bimodal_Fig5.png}
\end{figure}
\begin{figure}[htb!]
\includegraphics[width=\linewidth]{./Figs/Varied_Rank/bimodal_Fig6.png}
\end{figure}
\begin{figure}[htb!]
\includegraphics[width=\linewidth]{./Figs/Varied_Rank/bimodal_Fig7.png}
\end{figure}
\begin{figure}[htb!]
\includegraphics[width=\linewidth]{./Figs/Varied_Rank/bimodal_Recons.png}
\end{figure}


\end{document}  
