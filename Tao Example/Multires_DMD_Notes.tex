% !TEX TS-program = pdflatex
% !TEX encoding = UTF-8 Unicode

% This is a simple template for a LaTeX document using the "article" class.
% See "book", "report", "letter" for other types of document.

\documentclass[11pt]{article} % use larger type; default would be 10pt

\usepackage[utf8]{inputenc} % set input encoding (not needed with XeLaTeX)

%%% Examples of Article customizations
% These packages are optional, depending whether you want the features they provide.
% See the LaTeX Companion or other references for full information.

%%% PAGE DIMENSIONS
\usepackage{geometry} % to change the page dimensions
\geometry{a4paper} % or letterpaper (US) or a5paper or....
\geometry{margin=.7in} % for example, change the margins to 2 inches all round
% \geometry{landscape} % set up the page for landscape
%   read geometry.pdf for detailed page layout information

\usepackage{graphicx} % support the \includegraphics command and options

% \usepackage[parfill]{parskip} % Activate to begin paragraphs with an empty line rather than an indent

%%% PACKAGES
\usepackage{booktabs} % for much better looking tables
\usepackage{array} % for better arrays (eg matrices) in maths
\usepackage{paralist} % very flexible & customisable lists (eg. enumerate/itemize, etc.)
\usepackage{verbatim} % adds environment for commenting out blocks of text & for better verbatim
%\usepackage{subfig} % make it possible to include more than one captioned figure/table in a single float
% These packages are all incorporated in the memoir class to one degree or another...

\usepackage{amsmath}
\usepackage{amssymb}
\usepackage{bm}
%\usepackage{bbold}
\usepackage{bbm}
\usepackage{braket}
\usepackage{setspace}
\newcommand\id{\ensuremath{\mathbbm{1}}}

\usepackage{subcaption}

\begin{document}

\section{DMD Introduction}

Dynamic Mode Decomposition (DMD) takes a set of time series measurements governed by some non-linear equation $\dot{x} = f(x)$ and obtains a best-fit linear approximation of form $\dot{x} = Ax$. For a time series of $m$ measurements, this is accomplished by defining 
\begin{equation}
\begin{split}
X &= \left[\begin{matrix}
\vline & \vline & & \vline\\
x_1 & x_2 & \cdots & x_{m-1}\\
\vline & \vline & & \vline\\
\end{matrix}\right]\\
X' &= \left[\begin{matrix}
\vline & \vline & & \vline\\
x_2 & x_3 & \cdots & x_m\\
\vline & \vline & & \vline\\
\end{matrix}\right]\\
\end{split}
\end{equation}
and asserting the existence of a best-fit linear operator which maps $X$ onto $X'$:
\begin{equation}
\begin{split}
AX &\approx X'\\
A &= X'X^\dagger\\
\end{split}
\end{equation}
\textit{
(Note: For high-dimensional systems, $A$ is generally approximated using a rank-reduced version of $X$. Taking the singular value decomposition $X = U\Sigma V^*$ and omitting all but the $r << N$ most significant modes, one computes $\tilde{A} = U^* X' V \Sigma^{-1}$, which is an $r\times r$ projection of the full $A$. For this toy model, however, there are only $4$ input variables and therefore there is no need for this approximation)}\\

First-order linear ODEs of this form have solutions that can be written as simple exponentials in the eigenbasis of $A$. Specifically, the dynamics that result from the linear approximation operator $A$ take the form
\begin{equation}
\begin{split}
\label{dmd_recon_eq}
\bm{x}(t) = \bm{\Phi} \exp(\bm{\Omega} t) \bm{b}
\end{split}
\end{equation}
where $\bm{\Phi}$ is a matrix whose columns are eigenvectors of $A$ (in the measurement space of the input data), $\bm{\Omega}$ is a diagonal matrix of (complex) frequencies associated with those modes (logs of the eigenvalues of $A$), and $\bm{b}$ is a vector of weights obtained by projecting the measurement data onto these modes. Naively, $\bm{b}$ can simply be obtained by projecting the first snapshot $\bm{x}_1$ onto the eigenvectors:
\begin{equation}
\begin{split}
\bm{b} = \bm{\Phi}^\dagger \bm{x}_1
\end{split}
\end{equation}
This method gives arbitrary preference to the first measurement of the series, however, and as a result the DMD reconstructions given by Eq. \ref{dmd_recon_eq} tend to diverge from the true $X$ toward the end of the time series. A more principled approach known as Optimized DMD is used for the purposes of this study (details in arXiv:1704:02343v1 [math.NA]). 

\section{Windowed DMD}
The only time dynamics the above approach admits are exponential growth/decay (real $\omega$) and sinusoidal oscillation (imag. $\omega$). Similar to windowed Fourier or wavelet signal analysis, DMD is sensitive to the length of the input sample. If data is gathered over a duration of 10 seconds, it would of course be unreasonable to expect the algorithm to pick up on dynamics with characteristic time on the order of 1000 seconds (which would read as static in the measured window), or with characteristic time on the order of 0.001 seconds (which would read as noise).

This property is of essential importance for the analysis of systems with multi-resolution temporal dynamics. By varying the duration of the input time series, we can tune the algorithm to differentiate between the system's slow and fast dynamics. This is illustrated in Fig. \ref{fig:varied_window_size}. As the time series is divided into smaller and smaller windows, the decomposition does a better and better job of picking up on the fast dynamics. In the last plot, where windows are roughly 1 second in duration, we observe a clear bimodality in the frequency spectrum obtained with DMD.

%\begin{figure}[!t]
%\centering
%\subfloat[]{\includegraphics[width=1.1\linewidth]{dmd_window_1.png}}%
%\hfil
%\subfloat[]{\includegraphics[width=1.1\linewidth]{dmd_window_2.png}}%
%\hfil
%\subfloat[]{\includegraphics[width=1.1\linewidth]{dmd_window_3.png}}%
%\hfil
%\subfloat[]{\includegraphics[width=1.1\linewidth]{dmd_window_4.png}}%
%\caption{OptDMD results for different window sizes on the same data. On the right, measurement data (black) is overlaid with the DMD reconstruction from Eq. \ref{dmd_recon_eq} (colored). }
%\label{fig:varied_window_size}
%\end{figure}

\begin{figure}
\centering
\begin{subfigure}[b]{\textwidth}
	\includegraphics[width=1\linewidth]{dmd_window_1.pdf}
	\caption{7.2 second window size}
\end{subfigure}\\[.3 in]
\begin{subfigure}[b]{\textwidth}
	\includegraphics[width=1\linewidth]{dmd_window_2.pdf}
	\caption{4.1 second window size}
\end{subfigure}\\[.3 in]
\begin{subfigure}[b]{\textwidth}
	\includegraphics[width=1\linewidth]{dmd_window_3.pdf}
	\caption{2.0 second window size}
\end{subfigure}
\end{figure}
\begin{figure}\ContinuedFloat
\centering
\begin{subfigure}[b]{\textwidth}
	\includegraphics[width=1\linewidth]{dmd_window_4.pdf}
	\caption{1.0 second window size}
\end{subfigure}
\caption{OptDMD results for different window sizes on the same data. On the right, measurement data (black) is overlaid with the DMD reconstruction from Eq. \ref{dmd_recon_eq} (colored). The vertical dotted lines indicate how the time series has been divided. The DMD algorithm outlined above is applied independently to the data from each of the denoted windows. On the left, squared-frequency spectra are plotted for each window. In the center, all of those spectra are aggregated into a single histogram, which is fitted with a 2-component Gaussian Mixture Model (drawn in color over the histogram). This model is used to classify the DMD modes into high and low frequency categories (red and blue, respectively).}
\label{fig:varied_window_size}
\end{figure}

\section{Frequency Separation with DMD}
The data our model generates is purely real, but the reconstruction produced by DMD is generically complex. To fit to a real data set, the algorithm picks out either: 1) Real modes (with eigenvector $\bm{\phi}$ and frequency $\omega$ both purely real), or 2) Complex conjugate mode pairs (with eigenvectors $\bm{\phi}_i = \bm{\phi}_j^*$ and frequencies $\omega_i = \omega_j^*$). Either way, the full reconstruction given by Eq. \ref{dmd_recon_eq} is entirely real. The former case only permits growth/decay dynamics, whereas the latter can also accommodate sinusoidal oscillation.

The last plot of Fig. \ref{fig:varied_window_size} shows clear separation between high-frequency (HF) and low-frequency (LF) modes (clustered around $|\omega|^2 \approx 100$ and $|\omega|^2 \approx 1$, respectively). For this window size, the algorithm consistently identifies 2 HF modes which are conjugate to one another (which of course just sit on top of each other on the $|\omega|^2$ spectrum plot).

Having clustered the 4 modes of each window into 2 HF and 2 LF, we can now plot separate LF and HF reconstructions:
\begin{equation}
\begin{split}
\label{sep_recon_eq}
\bm{x}_{LF}(t) = \sum_{k\in\{LF\}} \bm{\phi}_k \exp(\omega_k t) b_k\\
\bm{x}_{HF}(t) = \sum_{k\in\{HF\}} \bm{\phi}_k \exp(\omega_k t) b_k\\
\end{split}
\end{equation}
The result of this is plotted in Fig. \ref{fig:freq_sep_recon}.

\begin{figure}
\centering
\includegraphics[width=0.9\linewidth]{sep_recon.pdf}
\caption{Top: The full  (all 4 modes) DMD reconstruction plotted over the measurement data. Bottom: HF and LF reconstructions (2 modes each) plotted separately for each window (from Eq. \ref{sep_recon_eq}). Because DMD was carried out for each 1-second window independently, the reconstructions do not necessarily match up at the boundaries and we obtain a discontinuous piecewise function. Note that each mode has components in all $4$ of the measurement variables, so the HF and LF reconstructions each still live in the full-dimensional space.}
\label{fig:freq_sep_recon}
\end{figure}

\section{HF Model Discovery}
Restricting our analysis for the time being to the HF modes, we seek to identify a model just for the dynamics in the fast time-scale regime. To do this, we first project the measurement data onto the HF modes obtained by windowed DMD. As discussed previously, these generally come in complex conjugate pairs. Projecting the measured $\bm{x}(t)$ onto such a pair results in two time series which are simply complex conjugates of one another---not only are the HF modes not orthogonal, in this example they can be treated as completely redundant! As far as DMD can discern, all HF dynamics occur at frequency $\omega = \text{Re}[\omega_1^{HF}] = \text{Re}[\omega_2^{HF}]$ on the mode $\bm{\phi} = \text{Re}[\bm{\phi}_1^{HF}] = \text{Re}[\bm{\phi}_2^{HF}]$. We proceed by projecting the measured $\bm{x}(t)$ onto this mode for each 1-second window.

Simple harmonic motion is given by $\ddot{x} = -\omega x$, which can be recast as a pair of first-order equations $\dot{x} = y$; $\dot{y} = -\omega x$. Using this as an intuitive guide, we differentiate the projection onto the HF mode and feed that derivative into the model-discovery algorithm as a separate variable. The projection and its derivative are plotted in Fig. \ref{fig:hf_proj}.

\begin{figure}
\centering
\includegraphics[width=0.9\linewidth]{hf_proj.pdf}
\caption{Top: Projection of measurement data onto the real HF mode. Bottom: Time derivative of this projection. Note that these are still discontinuous across the window boundaries, but the SINDy algorithm is known to generally be robust to data with a few such discontinuities.}
\label{fig:hf_proj}
\end{figure}

These variables are then fed into the Sparse Identification of Nonlinear Dynamics (SINDy) algorithm, which compares their time series to a library of possible functions for $\dot{\bm{x}} = f(\bm{x})$ and uses a sparse regression algorithm to identify the most likely model candidate.

The SINDy results are plotted in Fig. \ref{fig:hf_mode_sindy}. With the appropriate regression parameters, the algorithm was able to back out the true model quite accurately:
\begin{equation}
\begin{split}
\dot{y}_1 &= y_2\\
\dot{y}_2 &= -\epsilon^{-1}y_1
\end{split}
\end{equation}
The ground-truth $\epsilon$ was $0.01$, and SINDy identified it as $0.01005$.
\begin{figure}
\centering
\includegraphics[width=0.9\linewidth]{hf_mode_sindy.pdf}
\caption{Top: SINDy input, i.e. the projection of measured data onto the real HF mode for each window (red), and derivative of this projection (blue). Bottom: Dynamics from the model obtained by SINDy.}
\label{fig:hf_mode_sindy}
\end{figure}

\section{Caveats}
\subsection{SINDy Fine-Tuning}
The success of SINDy on this problem is encouraging, but I was only able to make it converge to the ``proper'' model for a particular narrow set of input parameters. The function library consisted of only polynomials up to 1st order, so it was only searching a space of 6 free parameters:
\begin{equation}
\begin{split}
\dot{y}_1 &= \alpha_1 y_1 + \beta_1 y_2 + \gamma_1\\
\dot{y}_2 &= \alpha_2 y_1 + \beta_2 y_2 + \gamma_2\\
\end{split}
\end{equation}
and even in this limited case it only zeroed out the appropriate parameters (i.e. $\alpha_1, \gamma_1, \beta_2, \gamma_2$) for a fairly narrow set of values for the sparseness tuning parameter $\lambda$ (see Fig. \ref{fig:lambda_tune}).
\begin{figure}
\centering
\includegraphics[width=0.9\linewidth]{lambda_tune.pdf}
\caption{Number of nonzero coefficients (out of a possible 6) identified by SINDy as a function of the sparse thresholding parameter $\lambda$}
\label{fig:lambda_tune}
\end{figure}

\subsection{Simple Harmonic Motion}
DMD is predisposed to pick out sinusoidal oscillation in the input data, so this procedure would likely prove less successful if the HF model were some nonlinear oscillator instead. In this case, the best DMD approximation would still be roughly sinusoidal (maybe with some exponential growth/decay component tacked on). After we've projected onto the HF modes it would be hard to identify the behavior as nonlinear, because it's already been reduced to its linear approximation.

This is a situation where a higher-dimensional system might actually prove a blessing: with more measurement variables, DMD will in a sense have more modes to work with. Instead of a single complex conjugate pair in the HF regime, there will be a more diverse and varied set of dynamical pieces with which to reconstruct something that more closely resembles the ``true'' nonlinearity.

\subsection{Non-Orthogonality of Modes}
DMD produces modes which are generically non-orthogonal. As a result, trying to separate out the time-scale regimes by simple projection is a somewhat dubious approach: some elements of the measurement data will inevitably be represented redundantly in both regimes. This doesn't mean that the projections are useless for analysis, but we will have to be careful about how we go about subtracting off HF dynamics from the measurement data, for instance. 

\end{document}
